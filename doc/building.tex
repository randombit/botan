\documentclass{article}

\setlength{\textwidth}{6.5in}
\setlength{\textheight}{9in}

\setlength{\headheight}{0in}
\setlength{\topmargin}{0in}
\setlength{\headsep}{0in}

\setlength{\oddsidemargin}{0in}
\setlength{\evensidemargin}{0in}

\title{\textbf{Botan Build Guide}}
\author{Jack Lloyd \\
        \texttt{lloyd@randombit.net}}
\date{}

\newcommand{\filename}[1]{\texttt{#1}}
\newcommand{\module}[1]{\texttt{#1}}

\newcommand{\type}[1]{\texttt{#1}}
\newcommand{\function}[1]{\textbf{#1}}
\newcommand{\macro}[1]{\texttt{#1}}

\begin{document}

\maketitle

\tableofcontents

\parskip=5pt
\pagebreak

\section{Introduction}

This document describes how to build Botan on Unix/POSIX and MS
Windows systems. The POSIX oriented descriptions should apply to most
common Unix systems (including MacOS X), along with POSIX-ish systems
like BeOS, QNX, and Plan 9. Currently, systems other than Windows and
POSIX (such as VMS, MacOS 9, OS/390, OS/400, ...) are not supported by
the build system, primarily due to lack of access. Please contact the
maintainer if you would like to build Botan on such a system.

\section{For the Impatient}

\begin{verbatim}
$ ./configure.pl
$ make
$ make install
\end{verbatim}

Or using \verb|nmake|, if you're compiling on Windows with Visual
C++. On platforms that do not understand the '\#!' convention for
beginning script files, or that have Perl installed in an unusual
spot, you might need to prefix the \texttt{configure.pl} command with
\texttt{perl} or \texttt{/path/to/perl}.

\section{Building the Library}

The first step is to run \filename{configure.pl}, which is a Perl
script that creates various directories, config files, and a Makefile
for building everything. The script requires at least Perl 5.6; any
later version should also work.

The script will attempt to guess what kind of system you are trying
to compile for (and will print messages telling you what it guessed).
You can override this process by passing the options \verb|--cc|,
\verb|--os|, and \verb|--arch| -- acceptable values are printed if
you run \verb|configure.pl| with \verb|--help|.

You can pass basically anything reasonable with \verb|--cpu|: the
script knows about a large number of different architectures, their
sub-models, and common aliases for them. The script does not display
all the possibilities in its help message because there are simply too
many entries. You should only select the 64-bit version of a CPU (such
as ``sparc64'' or ``mips64'') if your operating system knows how to
handle 64-bit object code -- a 32-bit kernel on a 64-bit CPU will
generally not like 64-bit code.

The script also knows about the various extension modules
available. You can enable one or more with the option
``\verb|--modules=MOD|'', where \verb|MOD| is some name that
identifies the extension (or a comma separated list of them). Modules
provide additional capabilities which require the use of APIs not
provided by ISO C/C++.

Not all OSes or CPUs have specific support in \filename{configure.pl}. If the
CPU architecture of your system isn't supported by \filename{configure.pl}, use
'generic'. This setting disables machine-specific optimization
flags. Similarly, setting OS to 'generic' disables things which depend greatly
on OS support (specifically, shared libraries).

However, it's impossible to guess which options to give to a system
compiler.  Thus, if you want to compile Botan with a compiler which
\filename{configure.pl} does not support, you will need to tell it how
that compiler works. This is done by adding a new file in the
directory \filename{misc/config/cc}; the existing files should put you
in the right direction.

The script tries to guess what kind of makefile to generate, and it
almost always guesses correctly (basically, Visual C++ uses NMAKE with
Windows commands, and everything else uses Unix make with POSIX
commands). Just in case, you can override it with
\verb|--make-style=somestyle|. The styles Botan currently knows about
are 'unix' (normal Unix makefiles), and 'nmake', the make variant
commonly used by Windows compilers. To add a new variant (eg, a build
script for VMS), you will need to create a new template file in
\filename{misc/config/makefile}.

\pagebreak

\subsection{POSIX / Unix}

The basic build procedure on Unix and Unix-like systems is:

\begin{verbatim}
   $ ./configure.pl --module-set=[unix|beos] --modules=<other mods>
   $ make
   # You may need to set your LD_LIBRARY_PATH or equivalent for ./check to run
   $ make check # optional, but a good idea
   $ make install
\end{verbatim}

This will probably default to using GCC, depending on what can be
found within your PATH.

The 'unix' module set should work on most POSIX/Unix systems out there
(including MacOS X), while the 'beos' module is specific to BeOS. While the two
sets share a number of modules, some normal Unix ones don't work on BeOS (in
particular, BeOS doesn't have a working \function{mmap} function), and BeOS has
a few extras just for it. The library will pick a default module set for you
based on the value of OS, so there is rarely a reason to specify that.

The \verb|make install| target has a default directory in which it
will install Botan (typically \verb|/usr/local|). You can override
this by using the \texttt{--prefix} argument to
\filename{configure.pl}, like so:

\verb|./configure.pl --prefix=/opt <other arguments>|

On Unix, the makefile has to decide who should own the files once they are
installed. By default, it uses \texttt{root:root}, but on some systems (for
example, MacOS X), there is no \texttt{root} group. Also, if you don't have
root access on the system you will want them to be installed owned by something
other than root (like yourself). You can override the defaults at install time
by setting the \texttt{OWNER} and \texttt{GROUP} variables from the command
line.

\verb|make OWNER=lloyd GROUP=users install|

On some systems shared libraries might not be immediately visible to the
runtime linker. For example, on Linux you may have to edit
\filename{/etc/ld.so.conf} and run \texttt{ldconfig} (as root) in order for new
shared libraries to be picked up by the linker. An alternative is to set your
\texttt{LD\_LIBRARY\_PATH} shell variable to include the directory that the
Botan libraries were installed into.

\subsection{MS Windows}

The situation is not much different here. We'll assume you're using Visual C++
(for Cygwin, the Unix instructions are probably more relevant). You need to
have a copy of Perl installed, and have both Perl and Visual C++ in your path.

\begin{verbatim}
   > perl configure.pl --cc=msvc --os=windows [--cpu=CPU] --module-set=win32
   > nmake
   > nmake check # optional, but recommended
\end{verbatim}

The configure script will include the 'win32' module set by default if
you pass \verb|--os=windows|. This module set includes a pair of
entropy sources for use on Windows; at some point in the future it
will also add support for high-resolution timers, mutexes for thread
safety, and other useful things.

For Win95 pre OSR2, the \verb|es_capi| module will not work, because
CryptoAPI didn't exist. All versions of NT4 lack the ToolHelp32
interface, which is how \verb|es_win32| does its slow polls, so a
version of the library built with that module will not load under
NT4. Later systems (98/ME/2000/XP) support both methods, so this
shouldn't be much of an issue.

Unfortunately, there currently isn't an install script usable on
Windows. Basically all you have to do is copy the newly created
\filename{libbotan.lib} to someplace where you can find it later (say,
\verb|C:\Botan\|). Then copy the entire \verb|include\botan| directory, which
was constructed when you built the library, into the same directory.

When building your applications, all you have to do is tell the compiler to
look for both include files and library files in \verb|C:\Botan|, and it will
find both.

\pagebreak

\subsection{Configuration Parameters}

There are some configuration parameters which you may want to tweak before
building the library. These can be found in \filename{config.h}. This file is
overwritten every time the configure script is run (and does not exist until
after you run the script for the first time).

Also included in \filename{config.h} are macros which are defined if one or
more extensions are available. All of them begin with \verb|BOTAN_EXT_|. For
example, if \verb|BOTAN_EXT_COMPRESSOR_BZIP2| is defined, then an application
using Botan can include \filename{<botan/bzip2.h>} and use the Bzip2 filters.

\macro{BOTAN\_MP\_WORD\_BITS}: This macro controls the size of the
words used for calculations with the MPI implementation in Botan. You
can choose 8, 16, 32, or 64, with 32 being the default. You can use 8,
16, or 32 bit words on any CPU, but the value should be set to the
same size as the CPU's registers for best performance. You can only
use 64-bit words if an assembly module (such as \module{mp\_ia32} or
\module{mp\_asm64}) is used. If the appropriate module is available,
64 bits are used, otherwise this is set to 32. Unless you are building
for a 8 or 16-bit CPU, this isn't worth messing with.

\macro{BOTAN\_VECTOR\_OVER\_ALLOCATE}: The memory container
\type{SecureVector} will over-allocate requests by this amount (in
elements). In several areas of the library, we grow a vector fairly often. By
over-allocating by a small amount, we don't have to do allocations as often
(which is good, because the allocators can be quite slow). If you \emph{really}
want to reduce memory usage, set it to 0. Otherwise, the default should be
perfectly fine.

\macro{BOTAN\_DEFAULT\_BUFFER\_SIZE}: This constant is used as the size of
buffers throughout Botan. A good rule of thumb would be to use the page size of
your machine. The default should be fine for most, if not all, purposes.

\macro{BOTAN\_GZIP\_OS\_CODE}: The OS code is included in the Gzip header when
compressing. The default is 255, which means 'Unknown'. You can look in RFC
1952 for the full list; the most common are Windows (0) and Unix (3). There is
also a Macintosh (7), but it probably makes more sense to use the Unix code on
OS X.

\subsection{Multiple Builds}

It may be useful to run multiple builds with different
configurations. Specify \verb|--build-dir=<dir>| to set up a build
environment in a different directory.

\subsection{Local Configuration}

You may want to do something peculiar with the configuration; to
support this there is a flag to \filename{configure.pl} called
\texttt{--local-config=<file>}. The contents of the file are inserted into
\filename{build/build.h} which is (indirectly) included into every
Botan header and source file.

\pagebreak

\section{Modules}

There are a fairly large number of modules included with Botan. Some
of these are extremely useful, while others are only necessary in very
unusual circumstances. The modules included with this release are:

\newcommand{\mod}[2]{\textbf{#1}: #2}

\begin{list}{$\cdot$}
  \item \mod{alloc\_mmap}{Allocates memory using memory mappings of temporary
         files. This means that if the OS swaps all or part of the application,
         the sensitive data will be swapped to where we can later clean it,
         rather than somewhere in the swap partition.}

  \item \mod{comp\_bzip2}{Enables an application to perform bzip2 compression
         and decompression using the library. Available on any system that has
         bzip2.}

  \item \mod{comp\_zlib}{Enables an application to perform zlib compression and
         decompression using the library. Available on any system that has
         zlib.}

  \item \mod{eng\_aep}{An engine that uses any available AEP accelerator card
         to speed up PK operations. You have to have the AEP drivers installed
         for this to link correctly, but you don't have to have a card
         installed - it will automatically be enabled if a card is detected at
         run time.}

  \item \mod{eng\_gmp}{An engine that uses GNU MP to speed up PK operations.
         GNU MP 4.1 or later is required.}

  \item \mod{eng\_ossl}{An engine that uses OpenSSL's BN library to speed up PK
         operations. OpenSSL 0.9.7 or later is required.}

  \item \mod{es\_beos}{An entropy source that uses BeOS-specific APIs to gather
         (hopefully unpredictable) data from the system.}

  \item \mod{es\_capi}{An entropy source that uses the Win32 CryptoAPI function
         \texttt{CryptGenRandom} to gather entropy. Supported on NT4, Win95
         OSR2, and all later Windows systems.}

  \item \mod{es\_egd}{An entropy source that accesses EGD (the entropy
         gathering daemon). Common on Unix systems that don't have
         \texttt{/dev/random}.}

  \item \mod{es\_ftw}{Gather entropy by reading files from a particular file
          tree. Usually used with \texttt{/proc}; most other file trees don't
          have sufficient variability over time to be useful.}

  \item \mod{es\_unix}{Gather entropy by running various Unix programs, like
          \texttt{arp} and \texttt{vmstat}, and reading their output in the
          hopes that at least some of it will be unpredictable to an attacker.}

  \item \mod{es\_win32}{Gather entropy by walking through various pieces of
          information about processes running on the system. Does not run on
          NT4, but should run on all other Win32 systems.}

  \item \mod{fd\_unix}{Let the users of \texttt{Pipe} perform I/O with Unix
         file descriptors in addition to \texttt{iostream} objects.}

  \item \mod{ml\_unix}{Add hooks for locking memory into RAM. Usually requires
         the application to run as \texttt{root} to actually work, but if the
         application is not allowed to call \texttt{mlock}, no harm results.}

  \item \mod{mp\_asm64}{Use inline assembly to access the multiply instruction
         available on some 64-bit CPUs. This module only runs on Alpha, AMD64,
         IA-64, MIPS64, and PowerPC-64. Typically PKI operations are several
         times as fast with this module than without.}

  \item \mod{mux\_pthr}{Add support for using \texttt{pthread} mutexes to
         lock internal data structures. Important if you are using threads
         with the library.}

  \item \mod{mux\_qt}{Add support for using Qt mutexes to lock internal data
         structures.}

  \item \mod{tm\_hard}{Use the contents of the CPU cycle counter when
         generating random bits to further randomize the results. Works on x86
         (Pentium and up), Alpha, and SPARCv9.}

  \item \mod{tm\_posix}{Use the POSIX realtime clock as a high-resolution
         timer.}

  \item \mod{tm\_unix}{Use the traditional Unix \texttt{gettimeofday} as a high
         resolution timer.}

  \item \mod{tm\_win32}{Use Win32's \texttt{GetSystemTimeAsFileTime} as a high
         resolution timer.}

\end{list}

\pagebreak

\section{Building Applications}

\subsection{Unix}

Botan usually links in several different system libraries (such as
\texttt{librt} and \texttt{libz}), depending on which modules are
configured at compile time. In many environments, particularly ones
using static libraries, an application has to link against the same
libraries as Botan for the linking step to succeed. But how does it
figure out what libraries it \emph{is} linked against?

The answer is to ask the \filename{botan-config} script. This
basically solves the same problem all the other \filename{*-config}
scripts solve, and in basically the same manner. At some point in the
future, a transition to \filename{pkg-config} will be made (as it's
less work, and has more features), but right now it doesn't exist on
most Unix systems, while a plain Bourne shell script will run fine on
anything.

There are 4 options:

\texttt{--prefix[=DIR]}: If no argument, print the prefix where Botan
is installed (such as \filename{/opt} or \filename{/usr/local}). If an
argument is specified, other options given with the same command will
execute as if Botan as actually installed at \filename{DIR} and not
where it really is; or at least where \filename{botan-config} thinks
it really is. I should mention that it

\texttt{--version}: Print the Botan version number.

\texttt{--cflags}: Print options that should be passed to the compiler
whenever a C++ file is compiled. Typically this is used for setting
include paths.

\texttt{--libs}: Print options for which libraries to link to (this includes
\texttt{-lbotan}).

Your \filename{Makefile} can run \filename{botan-config} and get the
options necessary for getting your application to compile and link,
regardless of whatever crazy libraries Botan might be linked against.

\subsection{MS Windows}

No special help exists for building applications on Windows. However,
given that typically Windows software is distributed as binaries, this
is less of a problem - only the developer needs to worry about it. As
long as they can remember where they installed Botan, they just have
to set the appropriate flags in their Makefile/project file.

\end{document}
